\documentclass[11pt]{article} 

\usepackage{amsmath} % required for math fuctnions
\usepackage{amssymb}
\usepackage{amsthm}
\usepackage{array} % used for table formatting
\usepackage{caption}
\usepackage{color}
\usepackage{colortbl}
\usepackage{enumitem} % used for item seperation
\usepackage{float} % used for placing float object
\usepackage{geometry} % used for page size and margins
\usepackage{graphicx} % used for graphics and figures
\usepackage{hyperref} % used for hyperlinks
\usepackage{makecell}
\usepackage{mdframed}   % For figure borders
%\usepackage{minted}
\usepackage{multirow}
\usepackage{ninecolors} % used for various colors in the specturm
\usepackage{outlines}
\usepackage{pdfcomment}
\usepackage{proposal}
\usepackage{tabularray}
\usepackage{textcomp}
\usepackage{tikz} % used to tabular array
\usetikzlibrary{arrows.meta,calc} % This is for tikzlibrat
% \usepackage[colorinlistoftodos]{todonotes}
\usepackage{soul}
\usepackage{vmargin}
\usepackage{wrapfig}
%\usepackage[table]{xcolor} 


\hypersetup{colorlinks=true,linkcolor=blue,urlcolor=blue}
\UseTblrLibrary{booktabs}
\usetikzlibrary{arrows, arrows.meta, backgrounds, fit, positioning, shapes, shapes.geometric}
\newtheorem{theorem}{Theorem}[section]
\newtheorem{lemma}[theorem]{Lemma}
 % for colors in the document and dvipsnames gives access to 68 colors
 % https://www.overleaf.com/learn/latex/Using_colors_in_LaTeX
\setpapersize{USletter} % Set paper and margins
\setmarginsrb{1in}{1in}{1in}{1in}{0pt}{0mm}{0pt}{0mm}

\title{
    \vspace{-45pt}
    \fontsize{15pt}{18pt}\selectfont
    \textcolor{FlyersRed}
    {\textbf{POSE Phase I}: Enabling an Open Ecosystem for Thermo-Fluid Computational Intelligence in Manufacturing and Energy Systems}
}
\date{}
\author{}


% Define M column type
\newcolumntype{M}[1]{>{\centering\arraybackslash}m{#1}}
\newcolumntype{R}[1]{>{\raggedleft\arraybackslash}m{#1}}

\newcommand{\CO}[1]{CO\textsubscript{#1}}




\begin{document}
\pagestyle{empty} % To remove page nubering
% \pagenumbering{gobble} % NO page numbers for research.gov
\vspace{-4\baselineskip}
\begin{center}
    \Large\textbf{\textcolor{FlyersRed}{PROJECT SUMMARY}}
\end{center}
\vspace{-1.4\baselineskip}


\section*{Overview}
\vspace{-3pt}
\noindent
This Phase-I research aims at a structured transition of FAME (https://github.com/neoceph/FAME), a machine learning (ML) augmented finite-volume based thermal CFD solver, into a sustainable, distributed Open-Source Ecosystem (OSE) for Additive Manufacturing (AM) and supercritical (s\CO{2})  researchers. FAME is available to researchers under an open-source license on GitHub and has been validated against NIST metal AM experiments \cite{aminPhysicsGuidedHeat2024}, attracting limited academic research users. The proposed OSE will provide targeted training and illustrative use cases, covering topics such as the design and analysis of AM processes and the production of s\CO{2}-based energy systems. Phase I activities will include: 1) formalizing the managing organization; 2) developing an effective governance model; 3) identifying and expanding the external developer and user community; and 4) designing the onboarding, outreach, licensing, and sustainability strategies for tool support. Tool development and support are led by Dr. Abdullah A. Amin and Dr. Andrew J. Schrader from the Mechanical and Aerospace Engineering Department at the University of Dayton. These researchers bring domain expertise in the design, optimization, and application of AM and Energy Systems, directly responding to the broader need for open, reproducible, and extensible infrastructure in manufacturing and energy research and development. This proposal focuses on building the foundation for a sustainable and community-driven OSE for manufacturing and energy research.
%
\vspace{-3pt}
\section*{Intellectual Merit}
\vspace{-3pt}
\noindent
This project will advance the FAME framework into a sustainable, community-driven open-source ecosystem (OSE) that supports high-fidelity thermal-fluid simulations for additive manufacturing (AM) and s\CO{2}-based thermal management. Emphasizing reproducibility, modularity, and extensibility, FAME integrates data-augmented modeling with ray-tracing-based radiative heat transfer to accurately capture beam-powder interactions and phase change dynamics key to understanding defect formation, microstructure evolution, and thermal system performance. Designed for heterogeneous CPU/GPU architectures, the framework supports scalable simulations and enables rapid iteration. By coupling physics-based solvers with modern ML libraries, it provides a rigorous platform to develop and validate AI/ML methods for surrogate modeling, real-time optimization, and intelligent process control. The project will formalize a flexible architectural design and foster a distributed contributor base to sustain long-term innovation. Integrated training resources will support the onboarding of new users, reducing entry barriers while enabling participation in advancing computational methods for intelligent manufacturing and energy systems.
\vspace{-3pt}

\section*{Context of OSE}
\vspace{-3pt}
\noindent
The long-term vision for the proposed Open-Source Ecosystem (OSE) around FAME is to create a sustainable, community-driven platform that enables collaborative development of advanced thermo-fluid computational tools for additive manufacturing (AM) and supercritical CO$_2$ (sCO$_2$) energy systems. Guiding principles include fostering inclusivity through open governance, ensuring reproducibility via standardized documentation and testing, promoting ethical AI/ML integration to mitigate biases in simulations, and prioritizing extensibility to support emerging applications like digital twins and real-time process optimization. This OSE will address critical societal and national needs, such as accelerating the adoption of sustainable manufacturing technologies to reduce energy consumption and waste in U.S. industries, and enhancing energy efficiency in sCO$_2$-based systems for clean power generation amid climate challenges. By providing open access to high-fidelity simulation tools, the OSE will bridge gaps in computational resources for underserved research communities, including small businesses and academic institutions, thereby bolstering U.S. competitiveness in advanced manufacturing and renewable energy sectors as outlined in national priorities like the CHIPS and Science Act.

Anticipated broader impacts include economic benefits through reduced experimental costs for AM part qualification, improved workforce readiness via integrated training modules that prepare diverse STEM talent for industry roles, and ethical advancements by embedding security and privacy best practices in open-source workflows. The OSE will promote STEM diversity by targeting outreach to underrepresented groups, such as through partnerships with minority-serving institutions, and contribute to societal goals like sustainable energy transitions by enabling faster innovation in defect-free AM components for aerospace, automotive, and biomedical applications.

\begin{enumerate}
    \item \textbf{Pointer to Existing Open-Source Product:} The product is available in \url{github.com/neoceph/FAME}.
    \item \textbf{Current Status:} FAME employs a finite volume method (FVM) development model integrated with a convolution hierarchical deep-learning neural network (C-HiDeNN) for thermal-fluid simulations, with testing focused on validation against experimental benchmarks like NIST AM datasets. Dissemination occurs through GitHub hosting, comprehensive documentation via ReadTheDocs (including installation guides using Anaconda and conda environments), and academic publications. The user base is currently limited to a small number of academic researchers (evidenced by initial downloads and feedback from early adopters in AM studies), with no formal releases yet published, indicating an early-stage but robust prototype. The contributor base consists primarily of the core development team (2-3 individuals based on commit history), with open invitations for pull requests to encourage expansion.
    \item \textbf{Problem Addressed and Novelty:} FAME addresses the challenge of accurately predicting melt pool dynamics, cooling rates, and defect formation in laser powder bed fusion (LPBF) AM processes, where process variabilities (e.g., laser power, scan speed) and high computational costs hinder part qualification and certification in critical sectors like aerospace and energy. Its novelty lies in a physics-guided heat source model calibrated via higher-order proper generalized decomposition (HOPGD) and ML techniques, simplifying complex thermal-fluid interactions (e.g., Marangoni flow, vaporization, keyhole formation) into efficient cylindrical representations while coupling with modern ML libraries for surrogate modeling and optimization—superior to proprietary tools or heuristic models that often overlook volumetric energy density correlations. Substantiating evidence includes validation against the NIST AM Bench 2022 Challenge, achieving $<5\%$ relative error in melt pool dimensions and $<20\%$ in cooling rates/time above melting for IN718 builds, demonstrating $30\text{-}50\%$ improvements in prediction accuracy over scaling law-based alternatives, as shown in blind experimental comparisons \cite{aminPhysicsGuidedHeat2024}.
    \item \textbf{Team Qualifications:} The team is well-qualified to lead this OSE transition, with Principal Investigator Dr. Abdullah A. Amin bringing over 5 years of expertise in physics-based modeling for LPBF AM, including development of FAME and related publications on heat source calibration for defect prediction. As a researcher in the Mechanical and Aerospace Engineering Department at the University of Dayton, Dr. Amin has presented on AM process modeling (e.g., using ANSYS Fluent for thermal simulations) and validated tools against NIST standards. Co-Principal Investigator Dr. Andrew J. Schrader, Assistant Professor and Director of the Dayton Thermal Applications (DaTA) Laboratory at the University of Dayton, contributes deep domain knowledge in thermal sciences and energy systems, with research on concentrated solar-thermal applications, sCO$_2$ systems, and innovative energy technologies. His synergistic activities include mentoring students in computational thermo-fluids and collaborating on open-source initiatives, ensuring the team's ability to formalize governance, engage communities, and sustain the OSE.
\end{enumerate}

\section*{Broader Impacts}
\vspace{-3pt}
\noindent
The FAME OSE ecosystem will democritize access to advanced thermal-fluid simulation and AI-assisted design tools across the manufacturing and energy sectors. A central aim is to train the next generation of researchers, engineers, and developers in open-source software development, digital twin technologies, and thermo-fluid modeling specific to manufacturing and energy. The project will offer annual hands-on workshops and summer schools focused on software sustainability and domain-specific applications. A targeted talent development initiative will engage institutions across various geographic regions through customized outreach and curriculum modules, helping to prepare a skilled, workforce-ready STEM pipeline. Research outcomes and practical use cases will be shared through peer-reviewed publications, conference tutorials, and freely accessible training materials hosted on the project's online platform and amplified through social media. By fostering a community grounded in open science, reproducibility, and collaborative development, this initiative will advance U.S. priorities in energy and manufacturing while ensuring the long-term sustainability of the FAME platform.


\section*{Risk Analysis/Security Plan}
\vspace{-3pt}
\noindent
Identify project-relevant risks, such as contributor attrition, security vulnerabilities in code, data privacy issues, or ethical concerns in AI outputs. Reference guidance from CISA/NSA on securing software supply chains and OpenSSF best practices (e.g., adopting SLSA for build integrity).
Mitigation strategies: Implement vulnerability scanning tools (e.g., during scoping audits), establish policies for patching (e.g., within 30 days of discovery), ensure data privacy via anonymization and compliance with GDPR-like standards, and maintain chain of custody through version control. Discuss Phase I activities to explore mechanisms for quality assurance, secure content integration (e.g., code reviews), identity/access management (e.g., multi-factor authentication for contributors), and safety/privacy risks (e.g., bias audits for models).
\vspace{-3pt}
\section*{Scoping Activities for Phase I}
\vspace{-3pt}
\noindent
Outline specific, actionable scoping and planning activities to assess the product's readiness for OSE transition, user base viability, and developer community potential. These will inform a potential Phase II proposal. Activities should be feasible within 1 year and \$300K budget.
\vspace{-3pt}

\noindent
Specific scoping activities for Phase I include:
\vspace{-3pt}
\begin{itemize}
    \item \textbf{Ecosystem Discovery:} Strategy to evaluate the technological landscape via literature reviews, competitor analyses (e.g., surveying similar projects like [cite examples]), and stakeholder interviews (e.g., 20-30 users/developers). Justify OSE approach (e.g., enables distributed innovation unlike closed systems). Outline methods to identify potential users (e.g., via surveys targeting industry sectors) and new developers (e.g., analyzing GitHub forks and contributions).
    \item \textbf{Organization and Governance:} Activities to identify models (e.g., benchmarking against Apache or Linux Foundation via case studies), licensing (e.g., evaluating MIT vs. GPL for compatibility), CI/CD infrastructure (e.g., assessing tools like GitHub Actions for asynchronous development), quality/security processes (e.g., drafting ethical guidelines), and sustainability methods (e.g., exploring funding models like donations). Include metrics (e.g., contributor retention rates >50\%).
    \item \textbf{Risk Analysis/Security:} Build on the plan above with scoping tasks like third-party security audits, privacy impact assessments, and workshops on secure development. Explore tools for identity management (e.g., OAuth) and custody chains (e.g., blockchain for provenance, if applicable).
    \item \textbf{Community Building:} Activities to engage users and developers, such as virtual workshops (e.g., 2-3 events with 50+ participants), hackathons (e.g., themed on product extensions), and competitions. Identify required capabilities (e.g., expertise in Python/ML for contributors) and mechanisms (e.g., online forums, research networks). Emphasize inclusivity (e.g., scholarships for underrepresented participants).
\end{itemize}


\section*{Community Outreach Plan}
\vspace{-3pt}
\noindent
Outline activities to engage intellectual content developers (e.g., webinars on contribution guidelines) and identify early adopters (e.g., partnerships with organizations via targeted emails and social media campaigns). Include timelines (e.g., quarterly outreach events) and diversity focus (e.g., collaborating with minority-serving institutions).
\vspace{-3pt}


\section*{Evaluation Plan}
\vspace{-3pt}
\noindent
Describe actionable metrics (e.g., \# of new contributors engaged [target: 20+], user adoption growth [target: 50\% increase], governance framework drafts completed) and assessment methods (e.g., pre/post surveys, analytics from repository tools, external mentor feedback). Include quarterly reviews to track progress and adjust plans.
\vspace{-3pt}

\section{Prior NSF Support}
The PI's have no prior NSF support.
\newpage
\bibliographystyle{unsrt}
\bibliography{reference}
\end{document}